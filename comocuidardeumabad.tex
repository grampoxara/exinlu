%
%
% EXINLU Nº0                    %{{{
%
%  ____  _  _  __  __ _  __    _  _ 
% (  __)( \/ )(  )(  ( \(  )  / )( \
%  ) _)  )  (  )( /    // (_/\) \/ (
% (____)(_/\_)(__)\_)__)\____/\____/
%
%
% E-zine destinada à livre exploração de todas esferas do fenômeno psiconáutico 
% e suas implicações sobre a natureza, cultura, sociedade e consciência humana.  
% Não possui orientação interpretativa oficial e destina-se apenas a criar um 
% espaço pra divulgação dos temas propostos e fomentar a pesquisa auxiliando na 
% documentação.
% 
% Licença "Creative Commons Attribution-NonCommercial-ShareAlike 3.0".
% http://creativecommons.org/licenses/by-nc-sa/3.0/br/legalcode
% %}}}
%
%
%    Vim comandos marca 'c  %{{{
%
% :setlocal nospell
%
% :set textwidth=128
% :set formatoptions=tcrqw
% :set wrapmargin=7
%
% :mkview! exinlu.vim
% :so exinlu.vim
% :setl noai nocin nosi inde=
% :gqap reformatar paragrafo
% :loadview :mkview
%
% %}}}
%
%
%    Preâmbulo marca 'p  %{{{
%
%  Classe (artigo, livro, carta, etc)
\documentclass[pdftex]{article}
%  internacionalização
\usepackage[brazilian]{babel}
%  Aceita teclados com ç
\usepackage[utf8]{inputenc}
%  Caracteres latinos
\usepackage{lmodern}
%  Tipografia fina
\usepackage[protrusion=true,expansion=true]{microtype}
%
% %}}}
%
%
\begin{document}
%
%
\title{Como cuidar de \textit{bads}.} %{{{ Resumo marca 'r
\author{EXINLU}
\date{21/dez/2012}
\maketitle
\renewcommand{\abstractname}{}
\abstract{
        Não era pra ser uma roleta. Era da boa, experientes e o clima estava 
        agradável. Mas acontece\ldots\\*
        E não existe nenhum curso preparatório, aprendemos com os erros e 
        tentamos auxiliar quem está por perto.\\
        \par
        \indent A seguir você encontra um guia de `primeiros socorros'.
        }
\newpage %}}}
%
%
\section*{O que não fazer}
\par
    \indent Imagine se todos arranjassem uma solução e tentassem ao mesmo tempo. Caos! 
    Tente manter a cabeça fria, olhar o problema do alto, longe do tumulto e do 
    calor. As pessoas reagem entrando em pânico juntas, querem apagar o incêndio 
    entrando em combustão também, querem apertar mais a situação pra
    caber dentro do que acham saudável e normal, empacotá-la em celofane de risos 
    satisfeitos e comportados.\\*
    Badtrip não se `curar', não é uma insanidade recalcada que precisa ser 
    resolvida, não é uma enxaqueca esperando dipirona. A
    pessoa se mergulhou de universos em que não detinha subsídio afetivo ou lógico 
    pra mantê-los reconhecível e manuseável na
    forma ortodoxa. Este equívoco não é o problema, depois vira epifania; o 
    desemparelhamento momentâneo pode causar avarias e
    degringolar em desespero. Este descompasso nos faz sentir como que pilotando de 
    um cockpit desconhecido e sem manual.\\*
    Este desarranjo nos faz sentir como que de volta num corpo bebê recém parido, 
    desconcertado com as luzes intensas e a respiração
    seca. Alguns engatinham, outros choram.\\*
    \indent É incomum lidarmos com situações excepcionais, não temos essa 
    malemolência, principalmente naquelas em que até as estratégicas
    básicas de enfrentamento precisam ser depuradas. Alterando a percepção você não 
    só altera o recorte do mundo externo que é
    feito.\\
\par
\noindent Por algum motivo está seguro que é melhor intervir em algo, que tem disposição e capacidade pra isso. Talvez tenha
visto algum perigo que os outros não perceberam, talvez seja fácil prestar algum auxílio sem grandes chances de perturbar.
Porém, lembre-se: O melhor pra geral é o melhor, na média, pra cada um.\\
\indent Tipicamente é uma badtrip psicológica, raras as vezes que a pessoa se machuca ou precisa de cuidados médicos, mas tenha sempre
em mente os riscos reais, tanto de overdose, quanto do ambiente.\\*
\indent Afaste a pessoa de possíveis situações de risco. Nada de correr pelado na rua, nem brincar com facas, nada de fogueiras
na sala, tão pouco aulas de tobogã. Esse será o maior trabalho e o mais pertinente. Assegure as melhores condições possíveis e
espere passar. Que vai passar é uma certeza, é uma das poucas certezas que temos. Pegue a posição mais confortável possível,
evite acidentes e aguarde.\\
\par
\indent Certamente está sendo uma experiência profunda, encantamento assombrado, gostoso e insólito se mesclam indistintos,
algumas pitadas de dor. Não dá pra fritar ovos sem romper a placenta. Sentimentos extremamente aflorados. Alguns até curtem essa
possibilidade de rever conceitos viscerais, mesmo correndo o risco de desmoronamento, submergem brocas petrolíferas e
reestruturam toda a economia. Outros se perdem nos controles novos, metem os pés pelas mãos com o que tem na cachola e chegam à
configurações exóticas. É difícil entender o que ela está realmente precisando, de certo nem ela sabe. Você pode estar vendo a
"solução", mas não tem como 'empurrá'-la no caminho, no máximo dá pra desviar algumas pedras e facilitar a caminhada. Cuide dos
pés, dê apoio e prática.\\
\noindent Se tiver muita gente em cima, não vá aglomerar mais. Parte do processo de convencer a pessoa de que está tudo bem é
tratá-la como se estivesse tudo bem.
Tente sempre localizar a pessoa, dizendo o que ela usou, lembrando onde ela está. Sem afobá-la, apenas situando-a.\\
\par
\indent Sempre passa. Só precisa esperar tempo suficiente. Essa é uma das poucas certezas que temos, sempre passa. Por mais que
tentemos, e tentamos muito, sempre passou. E tão certo quanto que o sol nascerá amanhã.
%
%
%
%
\end{document}
%
%
%
% vim comand autoload
% vim:set noai nocin nosi tw=79 wm=5 fo=tcwq fdm=marker: 
