%
%
% EXINLU Nº0
%
%  ____  _  _  __  __ _  __    _  _ 
% (  __)( \/ )(  )(  ( \(  )  / )( \
%  ) _)  )  (  )( /    // (_/\) \/ (
% (____)(_/\_)(__)\_)__)\____/\____/
%
%
% E-zine destinada à livre exploração de todas esferas do fenômeno 
% psiconáutico e suas implicações sobre a natureza, cultura, sociedade e 
% consciência humana.

% Não possui orientação interpretativa oficial e destina-se apenas a criar 
% um  espaço pra divulgação dos temas propostos e fomentar a pesquisa 
% auxiliando na documentação.
% 
% Licença "Creative Commons Attribution-NonCommercial-ShareAlike 3.0".
% http://creativecommons.org/licenses/by-nc-sa/3.0/br/legalcode
%
%
%
%    Vim comandos
%
% :setlocal nospell
%
% :set textwidth=72=colorcolumn
% :set formatoptions=tcrqw
% :set wrapmargin=3
%
% :mkview! exinlu.vim
% :so exinlu.vim
% :setl noai nocin nosi inde=
% :gqap reformatar paragrafo
% :loadview :mkview
% seleção visual + J = juntar linhas
% :set nolist
%
%
%
%
%    Preâmbulo
%
%
%  Preambulo compartilhado.
%
%  Classe (artigo, livro, carta, etc)
\documentclass[pdftex]{article}
%  internacionalização
\usepackage[brazilian]{babel}
%  Aceita teclados com ç
\usepackage[utf8]{inputenc}
%  Caracteres latinos
\usepackage{lmodern}
%  Tipografia fina
\usepackage[protrusion=true,expansion=true]{microtype}
%  Primeiro parágrafo indentado 
\usepackage{indentfirst}
%  Sem espaço adicional após .
\frenchspacing
%  texto em colunas
\usepackage{multicol}

%
%
\begin{document}
%
%
\pagenumbering{gobble}
\pagestyle{empty}
%
\title{EXINLU -- Terence McKenna}
\author{Shaman Kingston}
\date{21/12/2012}
\maketitle
\renewcommand{\abstractname}{}
%
\begin{abstract}
\noindent``\emph{Autor e explorador norte-americano, passou o último 
quarto de século da sua vida estudando as bases ontológicas do 
xamanismo e sua etnofarmacologia aplicada}''
\begin{flushright}wikipedia\end{flushright}
\end{abstract}
\newpage
%
%
\section*{\begin{flushright}Vida\end{flushright}}
Terence nasceu em uma pequena cidade do Colorado onde passava a maior 
parte do tempo explorando os fósseis que afloravam em abundância na 
superfície. Observar há quão pouco tempo a espécie humana habita a Terra 
através das suas aventuras paleontológicas foi o seu primeiro contato 
com as idéias de novidade e complexificação acelerante que mais tarde 
constituiriam uma de suas principais teorias, a \emph{Timewave Zero}.\\
\par
Sua vida é marcada por viagens, tanto geográficas quanto astrais através 
do uso de formas de expansão da consciência. Entrou primeiro em contato 
com a cultura psicodélica dos anos 60 através do livro \emph{Portas da 
Percepção} de Aldous Huxley, e mais tarde indo graduar-se em Berkeley, o 
epicentro do movimento contracultural. Lá, como a maior parte dos outros 
jovens da época, conheceu o ácido lisérgico e a \emph{cannabis sativa}. 
Porém, a sua curiosidade e ingenuidade levaram-lhe com dezenove anos a 
fumar Dimetiltriptamina oferecida de surpresa por um amigo. Esse momento 
marcou uma reviravolta em sua vida, quando percebeu que teria de dedicar 
o resto de sua existência a tentar entender esta dramática experiência 
indescritível.\\
\par
Deste instante em diante foram numerosas as viagens de McKenna. 
Percorreu o Oriente como traficante de haxixe, explorando as escolas de 
meditação e decepcionando-se com seus resultados comparados aos obtidos 
através de enteógenos até o momento em que teve de fugir da Interpol. 
Também esteve no Oriente Médio sem nenhum dinheiro e até mesmo nas 
desoladas ilhas africanas das Seicheles esperando uma namorada que lhe 
abandonou por um anão. Mas a viagem mais importante de sua vida, e em um 
certo sentido a única viagem da qual Terence nunca voltou, foi à La 
Chorrera na Colômbia com seu irmão Dennis McKenna e amigos de 
Berkeley.\\
\par
Na viagem descrita em \emph{True Hallucinations}, McKenna veio à América 
do Sul em busca da ayahuasca, que em sua época existia apenas como rumor 
em periódicos etnobotânicos. Além de tê-la encontrado, também entrou em 
contato com xamãs que faziam uso do cogumelo \emph{Psilocybe cubensis}, 
que tornou-se o novo foco da viagem.  McKenna relata uma série de 
eventos sobrenaturais, como aparições de UFOs, telepatia, teletransporte 
e coincidências extremamente improváveis. Em meio a esse caos, ele pôde 
receber da \emph{Logos} que fala através do sagrado \emph{teonanácatl} 
as informações que desenvolveria por todos anos a seguir. A trajetória 
de McKenna não foi uma busca contínua por conhecimentos, mas uma 
tentativa de assimilar a imensa quantidade de saberes já encontrados 
nesta curta expedição.\\
\par
McKenna lamentava não ter a coragem de fazer uma nova expedição, e 
considerava a sua carreira como uma covarde alternativa à continuidade 
de suas explorações.  Sofreu de violentas enxaquecas durante toda a 
vida, quando um dia sonhos estranhos até mesmo para Terence prenunciaram 
uma crise que indicou a existência de um tumor cancerígeno letal e 
irreversível em seu cérebro com causas exclusivamente genéticas. Durante 
o seu último ano encarnado na Terra, McKenna tinha pleno conhecimento 
que sua morte iria ocorrer a qualquer momento nos dias próximos, e aos 
54 anos pôde contemplar novamente a brevidade da existência humana.\\
\end{document}
%
%
%
% vim comand autoload
% vim:set noai nocin nosi list tw=72 cc=72 wm=3 fo=tcwq fdm=marker: 
