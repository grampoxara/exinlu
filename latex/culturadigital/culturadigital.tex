%
%
% EXINLU Nº0
%
%  ____  _  _  __  __ _  __    _  _ 
% (  __)( \/ )(  )(  ( \(  )  / )( \
%  ) _)  )  (  )( /    // (_/\) \/ (
% (____)(_/\_)(__)\_)__)\____/\____/
%
%
% E-zine destinada à livre exploração de todas esferas do fenômeno 
% psiconáutico e suas implicações sobre a natureza, cultura, sociedade e 
% consciência humana.

% Não possui orientação interpretativa oficial e destina-se apenas a criar 
% um  espaço pra divulgação dos temas propostos e fomentar a pesquisa 
% auxiliando na documentação.
% 
% Licença "Creative Commons Attribution-NonCommercial-ShareAlike 3.0".
% http://creativecommons.org/licenses/by-nc-sa/3.0/br/legalcode
%
%
%
%    Vim comandos
%
% :setlocal nospell
%
% :set textwidth=72=colorcolumn
% :set formatoptions=tcrqw
% :set wrapmargin=3
%
% :mkview! exinlu.vim
% :so exinlu.vim
% :setl noai nocin nosi inde=
% :gqap reformatar paragrafo
% :loadview :mkview
% seleção visual + J = juntar linhas
% :set nolist
%
%
%
%
%    Preâmbulo
%
%
%  Preambulo compartilhado.
%
%  Classe (artigo, livro, carta, etc)
\documentclass[pdftex]{article}
%  internacionalização
\usepackage[brazilian]{babel}
%  Aceita teclados com ç
\usepackage[utf8]{inputenc}
%  Caracteres latinos
\usepackage{lmodern}
%  Tipografia fina
\usepackage[protrusion=true,expansion=true]{microtype}
%  Primeiro parágrafo indentado 
\usepackage{indentfirst}
%  Sem espaço adicional após .
\frenchspacing
%  texto em colunas
\usepackage{multicol}

%
%
\begin{document}
%
%
\pagenumbering{gobble}
\pagestyle{empty}
%
\title{EXINLU \\ Cultura Digital}
\author{}
\date{}
\maketitle
\renewcommand{\abstractname}{}
%
\begin{abstract}
\noindent``\emph{Deve-se transformar a linguagem, e nesse sentido acho 
interessante o avanço tecnológico. Timothy leary seguiu o caminho certo, 
de instaurar a rebelião dentro desse espaço novo que é a cibercultura. É 
necessário encontrar uma nova linguagem que seja independente. Acredito 
que essa linguagem pode vir de uma absorção da loucura e da poesia em 
nossas vidas, de uma forma livre. A cibercultura é uma forma, entre 
outras, de se fazer isso}''
\begin{flushright}Claudio Willer\end{flushright}
\end{abstract}
\vfill
\begin{small}
Entrevista com Cláudio Prado para o projeto "Cultura Digital".\\
O original pode ser encontado no endereço:\\*
\begin{flushright}\url{http://culturadigital.br/}\end{flushright}
\end{small}
\newpage
%
%
\section*{\begin{flushright}O que é cultura digital?\end{flushright}}
\begin{multicols}{2} 
A cultura digital é a cultura do século XXI. É a nova compreensão de
praticamente tudo. O fantástico da cultura digital é que a tecnologia trouxe à
tona mudanças concretas, reais e muito práticas em relação a tudo que está
acontecendo no mundo, mas também reflexões conceituais muito amplas sobre o que
é a civilização e o que nós estamos fazendo aqui. A mitologia do século XXI é
desencadeada a partir do digital. Eu diria que o teórico que junta essas duas
coisas é o Timothy Leary, com a Política do êxtase. Não o ecstasy droga, a
política do êxtase. Ele escreve isso em plenos anos 1960. Isso me pirou na
época. Eu e o Gil roubávamos livros do Timothy Leary para distribuir para as
pessoas. Ele diz assim: o computador é o LSD do século XXI . Uma antevisão
muito interessante de tudo aquilo que vinha acontecendo com o digital no lado
prático, juntando essas duas correntes. Eu diria a você que existem duas
vertentes da cultura digital: uma prática, real, do software livre, de novas
percepções de como fazer as coisas, novas possibilidades de acesso, de troca,
de viabilização da diversidade, que era impedida porque não podia ser
distribuída no século XX, todas essas novas possibilidades extraordinárias. Por
outro lado, há uma coisa conceitual muito profunda, do papel do ser humano
sobre a terra, que se desencadeia numa compreensão muito mais séria de inúmeras
questões, entre elas a questão ecológica.\\
\end{multicols} 
\section*{\begin{flushright}Quais são os fatores que criam as condições para
    que, a partir de 2003, ou seja, no início do milênio, o Brasil consiga
produzir formação políticas de cultura digital?\end{flushright}} 
\begin{multicols}{2}
Certamente foi o mandato do Gilberto Gil no Ministério da Cultura que trouxe
essa abertura. Pessoalmente, o meu trabalho no Ministério da Cultura e a
possibilidade que o Gil e o Ministério da Cultura tiveram de abrir espaço para
a construção de uma política da qual se sabia muito pouco naquele momento. A
primeira questão importante foi a possibilidade de trabalhar dentro do
Ministério, no nível ministerial, de forma conjunta entre sociedade civil e
governo, o que foi uma coisa extremamente nova, extremamente interessante,
densa e rica, cheia de problemas, mas que trouxe a possibilidade de o Governo
vislumbrar a velocidade que o digital traz embutido nele. A grande questão do
digital é essencialmente a velocidade com que ele avança. Entre a tipografia e
a imprensa são 300 anos, entre o digital e o YouTube consagrado são 15 anos.
Esse descompasso da medida do tempo é brutal.\\
\end{multicols}
\section*{\begin{flushright}E como foi sua atuação pessoal?\end{flushright}}
\begin{multicols}{2}
Ah, era um hippie no Ministério da Cultura. Foi isso. De um lado o Gil abrindo
espaços para uma nova reflexão, do outro eu. Ele disse: “eu, Gilberto Gil,
ministro, trabalho para que governos não sejam mais necessários um dia”. Um
ministro de Estado falando esse tipo de coisa assim já mostra tudo\ldots{} Ele
disse isso em Tunis, durante a Cúpula Mundial da Sociedade da Informação. O
pensamento dos anos 1960 trazido para dentro do Governo é muito mais
revolucionário do que os ex-exilados que também estavam dentro dos governos,
embora eu não tenha nada contra os exilados políticos da esquerda. Apenas eles
acabaram resultando numa acomodação política do século XX , enquanto que eu
acho que o movimento hippie instalado procurou furar tudo isso, caminhando para
o XXI.\\
\end{multicols}
\section*{\begin{flushright}É hoje que os meios propiciam as vontades dos anos
60?\end{flushright}}
\begin{multicols}{2}
Houve um amadurecimento daquela questão toda. O movimento nos anos 1960 foi uma
coisa de uma explosão muito violenta, muito rápida. Ninguém entendia direito o
que estava acontecendo. A proposta, como dizia Timothy Leary, era “ligue-se,
sintonize e caía fora”, pular fora do sistema, construir outra possibilidade da
realidade. Mas a grande maioria desse pessoal entendeu logo em seguida que dava
para ganhar dinheiro com isso, e o negócio todo foi engolido, o sistema como
bom engolidor de tudo absorveu uma pancada de gente. Mas os conceitos
essenciais da discussão dos anos 1960 ficaram subjacentes a uma questão que
amadureceu e hoje aparece como a essência das questões do século XXI. Ou seja,
a questão da diversidade, da distribuição, de ecologia, a ideia da liberdade
profunda. Porque a liberdade é uma palavra que foi detonada no século XX. Bush
foi para o Iraque em nome da liberdade. Foi uma palavra que ficou sem sentido
nenhum. A liberdade que a gente falava nos anos 1960 era a liberdade de trepar,
contra o tabu sexual, que continua existindo. A cultura digital é uma cultura
pós-freudiana, porque a cultura freudiana continua toda ela envolvida na culpa
de Freud, na culpa do sexo.\\
\end{multicols}
\section*{\begin{flushright}Como é que foi vivenciar este processo?\end{flushright}}
\begin{multicols}{2}
O que aconteceu na prática foi que quando o Gil foi nomeado [2003] eu o
procurei. No primeiro dia eu não consegui falar, acabei encontrando com o Gil
aqui em São Paulo, no encontro de Mídias Táticas, onde estavam o John Perry
Barlow e o Richard Barbrook, que são duas figuras curiosamente opostas e
curiosamente revolucionárias. Eles estavam discutindo e brigando na mesa com o
Gil. O Gil convidou o Barlow e o pessoal da Mídia Tática havia convidado o
Barbrook. Eu fui assistir ao evento e falei com o Gil logo depois. Aquilo
explodiu para mim como uma parte do quebra-cabeça que começava a se juntar na
minha compreensão. Eu ia conversar com o Gil naquele dia sobre fazer alguma
coisa com música, onde música e não o business fosse o centro da história. O
codinome para mim daquilo era o Templo da Música, o lugar onde a música era
cultuada, era discutida e distribuída numa visão oposta da de uma gravadora.
Isso aproveitando as possibilidades todas que o digital trazia, as
possibilidades múltiplas de gravar e tudo mais. Mas ouvindo aquela conversa
ali, um monte de coisa começou a fazer sentido, inclusive a frase do Timothy
Leary sobre o computador. Quando eu ouvi aquela frase pela primeira vez, eu
achei, como muita gente, que ele estava louco. Durante o evento aquilo tudo se
fechou e deu um sentido muito profundo dessa possibilidade libertária. Leary
tem um gráfico extremamente interessante, no livro Chaos and cyberculture, do
desenvolvimento tecnológico versus a quantidade de bit de informação que você
recebe por dia. Então à medida que a tecnologia vai avançando a quantidade de
bits que se recebe por dia aumenta. O gráfico é uma curva que vai subindo até
ficar totalmente vertical. No entanto, é mais do que isso. O gráfico linear não
explica, não chega a expressar aquilo que está acontecendo nessa realidade
quântica. Não são mais três dimensões, existe uma outra dimensão filosófica. E
aí entra nas profundezas do Timothy Leary, que é uma loucura. É uma compreensão
de que nós temos que lidar daqui para frente com o caos. Não o caos
desestruturante, mas o caos estruturante. Ou seja, o máximo da velocidade é a
quietude. O estado meditativo é o resultado da velocidade máxima.\\
\end{multicols}
\section*{\begin{flushright}E como foi a conversa com Gil?\end{flushright}}
\begin{multicols}{2}
Imediatamente troquei a ideia. Disse a ele: “Gil, vamos pensar a questão do
digital como questão cultural.” Ainda estava focado na música, mas já era isso.
Aí eu peguei aquela moçada que eu tinha visto ali no bastidor do Mídia Tática e
convidei o pessoal para ir para a minha casa, conversar e discutir. O Gil
respondeu: “Eu tenho um projeto onde isso se encaixa perfeitamente, as BACs
[Bases de Apoio à Cultura].” E me encaminhou para conversar com o Roberto
Pinho, que pediu um mês, dizendo que teria dinheiro. Na prática foi isso, um
bando de gente que começou a conversar.\\
\end{multicols}
\section*{\begin{flushright}Quem era esse bando de gente?\end{flushright}}
\begin{multicols}{2}
Era a moçada que estava pensando e discutindo a questão digital, sobretudo o
software livre, que era a questão essencial que rodava por trás disso tudo. Era
uma compreensão menos voltada para a questão das artes e da cultura. Isso foi
uma coisa que apareceu depois e acabou inclusive criando atrito com o pessoal
do software livre, que, na minha compreensão, anda no caminho do século XX
filosoficamente. Tem um lado do software livre que acabou virando
fundamentalista. E o fundamentalismo é a grande doença do século XX. Mas eram
vários grupos. O Arca, que era mais ligado ao software livre propriamente dito,
o MetáFora, já estava trabalhando a ideia do MetaReciclagem. MetaReciclagem é
reciclar dentro de uma percepção quântica e não puramente material. Houve uma
enorme confusão justamente com essa questão de qual o limite do hardware e do
software. Essas coisas se confundem de uma forma fantástica. O hardware se
submete ao software em um determinado momento, depois inverte, e nesse
ping-pong de hardware e software foi que aconteceu a revolução toda.\\
\end{multicols}
\section*{\begin{flushright}E quando o Ministério entrou no processo 
verdadeiramente? \end{flushright}}
\begin{multicols}{2}
Demorou um ano e meio até realmente o Ministério entrar. E durante esse tempo
todo nós trabalhamos discutindo o que o Governo deveria e poderia fazer. Isso
se confundindo com o já fazendo, porque aí começaram a acontecer coisas. Eu
comecei a representar o Ministério da Cultura nos eventos de software livre,
nos eventos de inclusão digital do Governo. Comecei a falar e discutir essa
questões do ponto de vista do prisma cultural, e logo de cara descobri que
existia um vácuo fantástico nesse movimento do software livre e de cultura
livre, que ainda não existia enquanto tal. A cultura livre não estava colocada
do jeito que está hoje, mas que existia um espaço para ela. Existia um vácuo
enorme para uma liderança cultural, e o Gil se encaixava perfeitamente nesse
papel, inclusive nas discussões dentro do Governo. Então eu fui ao ITI
[Instituto Nacional de Tecnologia da Informação] falar com o Sérgio Amadeu.
Quando chego lá em nome do Ministério da Cultura, abre-se um espaço gigantesco.
Dentro do Ministério, se criaram duas grandes correntes do trabalho. Uma delas
era trazer o digital para o campo da cultura e da política. Esse trabalho era
conduzido através da agenda do Gil, que eu pautei muito antes de começar o
trabalho efetivo no Ministério. O outro trabalho foi com a Cultura Digital nos
Pontos de Cultura. A gente propôs a ideia do Kit Multimídia para o Célio
Turino, que estava coordenando os Pontos de Cultura, e ele rapidamente
compreendeu e aceitou. Então havia uma questão prática muito concreta e real,
de levar esses conceitos para as pontas, para a periferia brasileira, para a
molecada que estava espalhada nos Pontos de Cultura, conjugada a uma questão
mais conceitual.\\
\end{multicols}
\section*{\begin{flushright}Os Pontos de Cultura são justamente a 
    política que emerge no lugar das BACs\ldots \end{flushright}}
\begin{multicols}{2}
O insight do Gil estava absolutamente correto, de que o caminho era pelas BACs,
e a coisa se realizou de uma forma extremamente interessante. A polaridade de
uma discussão conceitual, filosófica, política, cultural, por um lado, e por
outro verificar como que as periferias brasileiras, como a molecada reagia à
internet nessa dimensão cultural. Foi isso que deu a visibilidade internacional
para essa história toda, através da capa da revista Wired. Não era simplesmente
um discurso, ainda que o discurso em si já seja fantástico. O pensamento do Gil
teve repercussões até dentro da UNESCO. A Convenção da Diversidade foi
inteirinha pautada por uma visão da diversidade que sem o digital não existiria
jamais. É impossível imaginar a diversidade a não ser pela sua fada madrinha
que é o digital, que possibilita a distribuição. O discurso ficaria totalmente
vazio se não houvesse essa compreensão prática. A periferia brasileira está
avançadíssima em relação à compreensão do digital. O digital age de forma
instantânea, há um fenômeno similar de compreensão do que é possível ser feito
na era pós-industrial, pós-trabalho. Esse pessoal aprende a fazer upload antes
de ouvir falar em download. A compreensão do up e down era uma proposta de
interatividade, de articulação e não de simplesmente baixar uma coisa para
consumo. Era uma compreensão política que dava uma dimensão de possibilidades
que esse pessoal tinha pela frente pela primeira vez.\\
\end{multicols}
\section*{\begin{flushright}Quais foram as principais
    dificuldades?\end{flushright}}
\begin{multicols}{2}
A burocracia, obviamente. Mas a burocracia brasileira é furável, ela é
contornável, não é rígida como no primeiro mundo. Se a nossa burocracia fosse
igual à burocracia europeia haveria uma travação total desse processo. Mesmo
assim, a burocracia travou a possibilidade real de se contratar exatamente o
que a gente estava fazendo. O que a gente estava construindo era incontratável,
se tivesse isso sido lá. Ao mesmo tempo havia um engajamento político,
militante, de um bando de gente que topou avançar sem essas garantias.\\
\end{multicols}
\section*{\begin{flushright}Pensando na continuidade do processo, existe 
um desafio de como recriar as institucionalidades que dêem conta
disso.\end{flushright}}
\begin{multicols}{2}
Isso é uma coisa que hoje precisaria ser discutida. Na verdade, a questão
essencial é o seguinte: nós não sabemos exatamente onde nós vamos chegar com
esse projeto. Ele é experimental, e a condição do governo fazer uma coisa
experimental, ou que não sabe exatamente aonde se quer chegar, é muito
complicada. A lei impede isso em nome de uma lisura, e se fosse ficar atento às
lisuras do Governo, a gente teria feito uma coisa burocrática. Não teria tido a
importância que teve, teria todo mundo recebido direitinho, estaria tudo
certinho e não teria acontecido nada. Essa é a essência da questão: a reforma
de como o Estado se comporta. Hoje, em nome de uma lisura que não acontece,
impede-se a possibilidade de fazer uma experiência de ponta. Ou seja, o Governo
não pode ser de ponta. O Governo é por natureza conservador. Ao governo não é
possível inovar. A não ser no nosso caso. Nós abrimos uma brecha. Mas essa
brecha nunca resultou na discussão necessária, do que é preciso existir para
que o Governo consiga inovar. Hoje a solução é dar prêmios. Foi a solução
encontrada. Mas o prêmio tem que ser para uma coisa que já foi feita, porque aí
você não precisa prestar contas desse dinheiro. É interessante prestar atenção
nisso porque justamente a velocidade do digital propõe um salto de tal ordem
que se o Governo não mudar completamente o seu jeito de pensar vai ficar a
reboque.\\
\end{multicols}
\section*{\begin{flushright}Uma outra dimensão que essa experiência 
trouxe é que a infraestrutura para a cultura digital exige outros elementos, há
necessidade de servidores, de \emph{hardwares}, de \emph{softwares}\ldots
\end{flushright}}
\begin{multicols}{2}
O que a gente descobriu é que um moleque – e quando falo moleque eu estou
falando do cidadão – que tiver plugado na outra ponta, ele é um usuário de
jamanta, na banda larga. Ele tem caminhões para circular todos os dias. Ele
começa a subir e descer foto. Daqui a pouco ele começa a subir seu vídeo, e
precisa de banda de verdade para fazer isso. Ele precisa ter onde armazenar
isso, toda essa condição de infra-estrutura de acesso, que não é simplesmente o
acesso, mas a possibilidade da troca, que o Google entendeu de forma
brilhante.\\
\end{multicols}
\section*{\begin{flushright}Não tem um desafio de construção de 
    alternativas públicas?\end{flushright}}
\begin{multicols}{2}
Se um Governo pensasse como o Google, ele estaria construindo isso. Se o Brasil
estivesse pensando como o Google, estaria lançando um satélite e oferecendo
conexão para todo mundo. E o Brasil também teria feito uma fábrica de chip
livre. Isso para mim é a grande sacada. O território do software livre pára no
chip proprietário. Ninguém ousou pensar no chip livre. O chip precisa de
escala, precisa de um bilhão. Não custa nada um chip, mas custa caro montar uma
fábrica de chip. Você precisa vender uma quantidade enorme...\\
\end{multicols}
\section*{\begin{flushright}O que, na sua opinião, configura ou 
    consittui esse Brasil tão receptivo a essa cultura emergente?\end{flushright}}
\begin{multicols}{2}
A essência da cultura brasileira é tropicalista. O tropicalismo não é uma
invenção, é uma constatação da nossa possibilidade miscigenada de enten- der as
coisas, de redirigir as coisas, que vêm de uma forma muito rápida. O componente
essencial da cultura brasileira é a alegria. Eu vejo na questão da alegria a
grande aposta brasileira, que também é a do digital. Tenta explicar o que é um
mutirão para um gringo, um francês, um alemão, um americano... as pessoas da
favela se juntam para construir a casa do vizinho, no fim de semana, em troca
de um churrasco e umas cervejas. Muitas vezes é ele mesmo que paga as cervejas.
Isso é impossível. A grande manifestação que vem das periferias, que vem da
favela é a expressão do carnaval, que é um enorme processo colaborativo. Tanto
o mutirão como o carnaval são processos em que as pessoas se juntam para
conseguir alguma coisa coletivamente. Eu vejo na questão brasileira uma
predisposição para os processos coletivos e colaborativos e uma visão pública.
A nossa grande salvação é não ter dado certo como o Primeiro Mundo. Fôssemos
nós de primeiro mundo, a Amazônia estaria devastada, pois foi assim como o
primeiro mundo se construiu, desvastando o meio ambiente e transformando em
energia e matéria-prima. Quando Charles de Gaulle vem aqui e diz “esse não é um
país sério” ele estava certíssimo. Isso aqui é um país alegre.\\
\end{multicols}
\section*{\begin{flushright}Ele teria dito outra coisa, també, que ``O 
    Brasil é o país do futuro e sempre será''\@. Parece que ele 
errou\ldots \end{flushright}}
\begin{multicols}{2}
Graças a essa invisibilidade que o Brasil emerge. Era impossível olhar para o
Brasil. Na verdade, as coisas que emergem no século XXI são graças à
invisibilidade que tiveram. A própria internet nasce porque ela era invisível.
Ninguém percebeu a internet como grande escala. O mundo corporativo não
enxergou a internet como possibilidade de negócio. O mundo regulatório não viu
a internet como ameaça para nada. Ninguém tentou regular e cooptar, e quando
ela aparece já é grande o suficiente para ser anárquica, caótica,
incontrolável. Agora vêm os sistemas corporativos e os sistemas regulatórios,
governamentais, correr atrás do prejuízo. Aqui no Brasil essa coisa se tornou
curiosamente desgovernada. Essa possibilidade que temos de fazer arranjos e
acomodações que em outros lugares do mundo são impossíveis de acontecer. Quem
diz isso é o Barlow. Ele, quando olhou para nós aqui, quando se virou para o
que estava acontecendo aqui, disse: “Vocês são a possibilidade de a revolução
digital funcionar e dar certo”. O resto do mundo já conseguiu controlar isso de
alguma forma e aqui continua solto.\\
\end{multicols}
\section*{\begin{flushright}Quais são, na sua opinião, os principais 
    desafios para seguir adiante?\end{flushright}}
\begin{multicols}{2}
Nesse momento precisamos levar essa discussão ao extremo que ela pode ser
levada. Não deixar virar uma coisa morna. Nesse sentido é muito importante o
reconhecimento da importância internacional do Brasil. O papel dos
observadores, dos colaboradores internacionais, é extremamente interessante. E,
sobretudo, eu imagino que precisamos acompanhar a emergência dos políticos
digitais, o que vai acontecer de uma forma ou de outra, e, na minha percepção,
isso vai acontecer nos municípios, nas cidades digitais, que vão poder dar
saltos inacreditáveis. Uma administração municipal que entenda o que o digital
pode fazer para mudar as realidades locais, revolucionar as realidades locais,
e injetar auto-estima conseguirá muita coisa. Com autoestima levantada as
pessoas vão longe. O perigo agora é isso ser cooptado pelo sistema. Por outro
lado, as pessoas estão vindo. Existem milhares de demonstrações disso. O que
precisamos fazer é lutar para que a cultura livre, digital, multimídia chegue
para todos.\\
\end{multicols}
\end{document}
%
%
%
% vim comand autoload

