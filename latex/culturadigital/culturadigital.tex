%
%
% EXINLU Nº0                    %{{{
%
%  ____  _  _  __  __ _  __    _  _ 
% (  __)( \/ )(  )(  ( \(  )  / )( \
%  ) _)  )  (  )( /    // (_/\) \/ (
% (____)(_/\_)(__)\_)__)\____/\____/
%
%
% E-zine destinada à livre exploração de todas esferas do fenômeno 
% psiconáutico e suas implicações sobre a natureza, cultura, sociedade e 
% consciência humana.

% Não possui orientação interpretativa oficial e destina-se apenas a criar 
% um  espaço pra divulgação dos temas propostos e fomentar a pesquisa 
% auxiliando na documentação.
% 
% Licença "Creative Commons Attribution-NonCommercial-ShareAlike 3.0".
% http://creativecommons.org/licenses/by-nc-sa/3.0/br/legalcode
% %}}}
%
%
%    Vim comandos marca 'c  %{{{
%
% :setlocal nospell
%
% :set textwidth=72=colorcolumn
% :set formatoptions=tcrqw
% :set wrapmargin=3
%
% :mkview! exinlu.vim
% :so exinlu.vim
% :setl noai nocin nosi inde=
% :gqap reformatar paragrafo
% :loadview :mkview
% seleção visual + J = juntar linhas
% :set nolist
%
% %}}}
%
%
%    Preâmbulo
%
%
%  Preambulo compartilhado.
%
%  Classe (artigo, livro, carta, etc)
\documentclass[pdftex]{article}
%  internacionalização
\usepackage[brazilian]{babel}
%  Aceita teclados com ç
\usepackage[utf8]{inputenc}
%  Caracteres latinos
\usepackage{lmodern}
%  Tipografia fina
\usepackage[protrusion=true,expansion=true]{microtype}
%  Primeiro parágrafo indentado 
\usepackage{indentfirst}
%  Sem espaço adicional após .
\frenchspacing
%  texto em colunas
\usepackage{multicol}
%  hyperlinks
\usepackage{hyperref}

%
%
\begin{document}
%
%
\pagenumbering{gobble}
\pagestyle{empty}
%
\title{EXINLU \\ Cultura Digital} %{{{ Resumo marca 'r
\author{}
\date{}
\maketitle
\renewcommand{\abstractname}{}
%
\begin{abstract}
\noindent``\emph{Deve-se transformar a linguagem, e nesse sentido acho 
interessante o avanço tecnológico. Timothy leary seguiu o caminho certo, 
de instaurar a rebelião dentro desse espaço novo que é a cibercultura. É 
necessário encontrar uma nova linguagem que seja independente. Acredito 
que essa linguagem pode vir de uma absorção da loucura e da poesia em 
nossas vidas, de uma forma livre. A cibercultura é uma forma, entre 
outras, de se fazer isso}''
\begin{flushright}Claudio Willer\end{flushright}
\end{abstract}
\vfill
\begin{small}
Entrevista com Cláudio Prado para o projeto "Cultura Digital".\\
O original pode ser encontado no endereço:\\*
\begin{flushright}\url{http://culturadigital.br/}\end{flushright}
\end{small}
\newpage %}}}
%
%
\section*{\begin{flushright}O que é cultura digital?\end{flushright}}
\begin{multicols}{2} 
\end{multicols} 
\section*{\begin{flushright}Quais são os fatores que criam as condições 
para que, a partir de 2003, ou seja, no início do milênio, o Brasil 
consiga produzir formação políticas de cultura digital?\end{flushright}} 
\begin{multicols}{2}
\end{multicols}
\section*{\begin{flushright}E como foi sua atuação pessoal?\end{flushright}}
\begin{multicols}{2}
\end{multicols}
\section*{\begin{flushright}É hoje que os meios propiciam as vontades 
    dos anos 60?\end{flushright}}
\begin{multicols}{2}
\end{multicols}
\section*{\begin{flushright}Como é que foi vivenciar este processo?\end{flushright}}
\begin{multicols}{2}
\end{multicols}
\section*{\begin{flushright}E como foi a conversa com Gil?\end{flushright}}
\begin{multicols}{2}
\end{multicols}
\section*{\begin{flushright}Quem era esse bando de gente?\end{flushright}}
\begin{multicols}{2}
\end{multicols}
\section*{\begin{flushright}E quando o Ministério entrou no processo 
    verdadeiramente?\end{flushright}}
\begin{multicols}{2}
\end{multicols}
\section*{\begin{flushright}Os Pontos de Cultura são justamente a 
    política que emerge no lugar das BACs\ldots \end{flushright}}
\begin{multicols}{2}
\end{multicols}
\section*{\begin{flushright}Quais foram as principais dificuldades?\end{flushright}}
\begin{multicols}{2}
\end{multicols}
\section*{\begin{flushright}Pensando na continuidade do processo, existe 
um desafio de como recriar as institucionalidades que dêem conta disso.\end{flushright}}
\begin{multicols}{2}
\end{multicols}
\section*{\begin{flushright}Uma outra dimensão que essa experiência 
    trouxe é que a infraestrutura para a cultura digital exige outros 
elementos, há necessidade de servidores, de \emph{hardwares}, de 
\emph{softwares}\ldots \end{flushright}}
\begin{multicols}{2}
\end{multicols}
\section*{\begin{flushright}Não tem um desafio de construção de 
    alternativas públicas?\end{flushright}}
\begin{multicols}{2}
\end{multicols}
\section*{\begin{flushright}O que, na sua opinião, configura ou 
    consittui esse Brasil tão receptivo a essa cultura emergente?\end{flushright}}
\begin{multicols}{2}
\end{multicols}
\section*{\begin{flushright}Ele teria dito outra coisa, també, que ``O 
    Brasil é o país do futuro e sempre será''\@. Parece que ele 
errou\ldots \end{flushright}}
\begin{multicols}{2}
\end{multicols}
\section*{\begin{flushright}Quais são, na sua opinião, os principais 
    desafios para seguir adiante?\end{flushright}}
\begin{multicols}{2}
\end{multicols}





\end{document}
%
%
%
% vim comand autoload

