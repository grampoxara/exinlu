%
%
% EXINLU Nº0
%
%  ____  _  _  __  __ _  __    _  _ 
% (  __)( \/ )(  )(  ( \(  )  / )( \
%  ) _)  )  (  )( /    // (_/\) \/ (
% (____)(_/\_)(__)\_)__)\____/\____/
%
%
% E-zine destinada à livre exploração de todas esferas do fenômeno 
% psiconáutico e suas implicações sobre a natureza, cultura, sociedade e 
% consciência humana.

% Não possui orientação interpretativa oficial e destina-se apenas a criar 
% um  espaço pra divulgação dos temas propostos e fomentar a pesquisa 
% auxiliando na documentação.
% 
% Licença "Creative Commons Attribution-NonCommercial-ShareAlike 3.0".
% http://creativecommons.org/licenses/by-nc-sa/3.0/br/legalcode
%
%
%
%    Vim comandos
%
% :setlocal nospell
%
% :set textwidth=72=colorcolumn
% :set formatoptions=tcrqw
% :set wrapmargin=3
%
% :mkview! exinlu.vim
% :so exinlu.vim
% :setl noai nocin nosi inde=
% :gqap reformatar paragrafo
% :loadview :mkview
% seleção visual + J = juntar linhas
% :set nolist
%
%
%
%
%    Preâmbulo
%
%
%  Preambulo compartilhado.
%
%  Classe (artigo, livro, carta, etc)
\documentclass[pdftex]{article}
%  internacionalização
\usepackage[brazilian]{babel}
%  Aceita teclados com ç
\usepackage[utf8]{inputenc}
%  Caracteres latinos
\usepackage{lmodern}
%  Tipografia fina
\usepackage[protrusion=true,expansion=true]{microtype}
%  Primeiro parágrafo indentado 
\usepackage{indentfirst}
%  Sem espaço adicional após .
\frenchspacing
%  texto em colunas
\usepackage{multicol}

\usepackage{enumitem}
%
%
\begin{document}
%
%
\pagenumbering{gobble}
\pagestyle{empty}
%
\title{}
\author{}
\date{}
\renewcommand{\abstractname}{}
%
\section*{Argyreia}
\subsection*{\begin{flushright}Introdução\end{flushright}}
\emph{Argyreia nervosa} é uma trepadeira de porte avantajado e 
semi-lenhosa, de raízes profundas e crescimento vigoroso, cada rama pode 
chegar a 10 metros!\\
É uma planta tropical, necessita de muito sol e muita umidade. Seu 
crescimento vigoroso será penalizado se não houver nutrientes 
em abundância.
\subsection*{\begin{flushright}Germinação\end{flushright}}
Em cada semente há um ``olho''. É circular, esbranquiçado e de aparência 
mais frágil que o resto da casca marrom. Plante com esta parte para 
baixo.
Para quebrar a dormência deixe por uma noite de molho. Pra ajudar com uma 
semi-cesariana você pode abrir um pequeno corte ou furo no lado oposto ao 
olho, é onde as folhas embrionárias arrebentam a casca.\\
Se for plantar onde a mudinha corra muito risco antes de crescer 
(parques e outros lugares públicos, ou se o teu cachorro gostar de cavar 
o quintal), plante em vaso e depois faça o transplante, ela é bem 
tolerante pois rápido cresce e repõem as partes danificadas. Mas não use 
vasos pequenos, o tanto ela esticar pro alto é o quanto ela precisa de 
vaso.\\
\subsection*{\begin{flushright}Crescimento\end{flushright}}
O primeiro par de folhas já estão dentro da semente, desabrocham na 
terra descolando-se uma da outra. Do meio brota a primeira rama. Depois 
de um mês ela está com umas 4 a 6 folhas e subindo, ai começa a girar 
buscando apoio. Depois ela começa a enrijecer o `caule', não deixe 
passar desta fase pra replantar.\\
Se a planta tiver oportunidade de crescer se pendurando verticalmente, 
escalando pro céu, ela não vai soltar ramas laterais até encontrar um 
obstáculo.\\
\newpage
\subsection*{\begin{flushright}Dicas gerais\end{flushright}}
Tenho uma Argyreia que deu sementes em menos de 6 meses e em pleno 
inverno, faço duas suposições:\\*
\begin{description}
    \item [Fotoperiodismo] A Argyreia está na grade de casa e perto do 
poste de luz. A luminosidade está enganando-a, fazendo-a funcionar a noite 
como se fosse de dia.
    \item [Competição] Existem 3 bem juntas. Uma encosta na outra e enrola. Às 
vezes até se enforcam uma na outra.
\end{description}
\par
\indent Faça um composteira pra reciclar o lixo orgânico que iria fora. As 
folhas, flores e ramos mortos da planta são excelentes pra compostagem,
possuem todos os nutrientes necessários pra planta nas doses corretas
(porque é a própria planta!).\\*
http://casa.hsw.uol.com.br/faca-sua-composteira.htm\\*
Pense em toda a massa que está surgindo na tua cerca, ela precisa sair
de algum lugar (do sequestro do carbono e da matéria orgânica do
substrato). Se for usar adubos, prefira com macronutrientes e de plantas 
com crescimento rápido (hortaliças/samambaias).\\
\par
\indent Quando a rama for suficientemente grande pra ser arcada, enrole-a na 
própria planta ou grade deixando a barriga para cima ($\frown$). No topo 
brotará um novo galho mirando pro alto. Deixe o galho crescer e o curve 
novamente. Repita isso até a planta ficar parecida com o cabelo da marge 
simpson. Forçar o crescimento de brotos em lugares específicos faz com 
que a planta gere menos brotos em locais suicida\\*
A propagação com alporquia é muito mais rápido, é o clone de uma planta 
adulta e saudável. Se um galho num lugar inapropriado, faça uma muda 
dele.\\
\par
\indent Mantenha a porção de solo perto da planta bem porosa, cheia de 
galhos, folhas, restos de frutos, etc. Isto faz com que a água não 
escorra, permaneça mais tempo, penetrando mais.\\*
Em tempo de seca a rega acaba sendo mais superficial, não queremos 
gastar água pra encharcar o solo todo. Com este retensor natural você pode 
ligar menos e deixar mais tempo apenas perto da planta. Ao invés de 
espalhar sobre o solo, a água infiltra e alcança regiões mais fundas das 
raízes, mantendo o crescimento delas para baixo em busca de água.\\
\hfill
\begin{flushright}Propague esta ideia ;-)\end{flushright}

\end{document}
%
%
%
% vim comand autoload
% vim:set noai nocin nosi list tw=72 cc=72 wm=3 fo=tcwq fdm=marker: 
