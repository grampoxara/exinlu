%
%
% EXINLU Nº0                    %{{{
%
%  ____  _  _  __  __ _  __    _  _ 
% (  __)( \/ )(  )(  ( \(  )  / )( \
%  ) _)  )  (  )( /    // (_/\) \/ (
% (____)(_/\_)(__)\_)__)\____/\____/
%
%
% E-zine destinada à livre exploração de todas esferas do fenômeno psiconáutico 
% e suas implicações sobre a natureza, cultura, sociedade e consciência humana.  
% Não possui orientação interpretativa oficial e destina-se apenas a criar um 
% espaço pra divulgação dos temas propostos e fomentar a pesquisa auxiliando na 
% documentação.
% 
% Licença "Creative Commons Attribution-NonCommercial-ShareAlike 3.0".
% http://creativecommons.org/licenses/by-nc-sa/3.0/br/legalcode
% %}}}
%
%
%    Vim comandos marca 'c  %{{{
%
% :setlocal nospell
%
% :set textwidth=72=colorcolumn
% :set formatoptions=tcrqw
% :set wrapmargin=3
%
% :mkview! exinlu.vim
% :so exinlu.vim
% :setl noai nocin nosi inde=
% :gqap reformatar paragrafo
% :loadview :mkview
% seleção visual + J = juntar linhas
% :set nolist
%
% %}}}
%
%
%    Preâmbulo
%
%
%  Preambulo compartilhado.
%
%  Classe (artigo, livro, carta, etc)
\documentclass[pdftex]{article}
%  internacionalização
\usepackage[brazilian]{babel}
%  Aceita teclados com ç
\usepackage[utf8]{inputenc}
%  Caracteres latinos
\usepackage{lmodern}
%  Tipografia fina
\usepackage[protrusion=true,expansion=true]{microtype}
%  Primeiro parágrafo indentado 
\usepackage{indentfirst}
%  Sem espaço adicional após .
\frenchspacing
%  texto em colunas
\usepackage{multicol}
%  hyperlinks
\usepackage{hyperref}

%
%
\begin{document}
%
%
\pagenumbering{gobble}
\pagestyle{empty}
%
\title{EXINLU \\ Como cuidar de \textit{bads}} %{{{ Resumo marca 'r
\author{}
\date{}
\maketitle
\renewcommand{\abstractname}{}
\begin{abstract}
\emph{Não era pra ser uma roleta. Era da boa, experientes e o clima estava 
agradável. Mas acontece\ldots}\\*
E não existe nenhum curso preparatório, aprendemos com os erros e 
tentamos auxiliar quem está por perto.\\
\par
A seguir você encontra um guia de `primeiros socorros'.
\end{abstract}
\newpage %}}}
%
%
\section*{\begin{flushright}O que não fazer\end{flushright}}
\par
\begin{multicols}{2} 
Imagine se todos arranjassem uma solução e tentassem ajudar o sujeito, 
imagine qual seria a solução que os intrépidos acompanhantes, em suas 
mentes beligerantes e apresuntadas, iriam conseguir apanhar. Imagine 
como todos em volta articulariam uma tática conjunta e sensata. Caos!\\* 
Tente primeiro manter a cabeça fria, olhar o problema do alto, longe do 
tumulto e do calor. As pessoas reagem entrando em pânico juntas, querem 
apagar o incêndio com querosene e um pouco de malícia, querem espremer 
mais a situação pra caber dentro do que acham saudável e normal, 
empacotá-la em celofane de risos satisfeitos e comportados. Não conte 
com a pessoa explicando bem qual é o problema, provavelmente ela tem a 
menor menor noção (muito menos acredite que alguém conseguirá entender o 
que ela disse).\\ 
\indent A realidade é muito pouco confiável nestes momentos, nunca se 
sabe se o 6 não era 9. Por isso qualquer possibilidade de piora deve ser 
evitada a todo custo. Aguentar firme e esperar pacientemente deveria ser 
a reação típica, qualquer mudança pode ser terrivelmente inapropriada, 
não se sabe qual será a reação.\\ 
\par 
Tipicamente é uma badtrip psicológica, raras as vezes que a pessoa se 
machuca ou precisa de cuidados médicos, mas tenha sempre em mente os 
riscos reais, tanto de overdose, quanto do ambiente. Afaste a pessoa de 
possíveis situações de risco. Nada de correr pelado na rua, nem brincar 
com facas, nada de fogueiras na sala, tão pouco aulas de tobogã. Esse 
será o maior trabalho e o mais pertinente. Assegure as melhores 
condições possíveis e espere passar. Que vai passar é uma certeza, é uma 
das poucas certezas que temos. 
\end{multicols} 
\section*{\begin{flushright}Tormenta -- Olhos Galático\end{flushright}} 
\par 
\begin{multicols}{2} 
\indent Certamente está sendo uma experiência profunda, encantamento 
assombrado, gostoso e insólito se mesclam indistintos. Algumas pitadas 
de dor, não dá pra fritar ovos sem romper a placenta. Sentimentos 
extremamente aflorados. Alguns até curtem essa possibilidade de rever 
conceitos viscerais, mesmo correndo o risco de desmoronamento, submergem 
brocas petrolríferas e reestruturam toda a economia. Outros se perdem nos 
controles novos, metem os pés pelas mãos com o que tem na cachola e 
chegam à configurações exóticas. Você pode estar vendo a ``solução'', 
mas não tem como `empurrá'-la no caminho, no máximo dá pra desviar 
algumas pedras e facilitar a caminhada.\\ 
\par 
Badtrip não se `curar', não é uma insanidade recalcada que precisa ser 
resolvida, não é uma enxaqueca esperando dipirona. A pessoa se mergulhou 
de universos em que não detinha subsídio afetivo ou lógico pra mantê-los 
reconhecível e manuseável na forma ortodoxa, na forma em que sabia 
manusear. Os inúmeros reflexos virtuais -- o peso e a inércia refletindo 
nosso volume, o espelho refletindo nossa aparência, a memória refletindo 
nossa história -- compondo uma unidade reconhecível como `EU', estes 
reflexos também são percebidos de forma alterada, rebocada, cravejado de 
rodopios invasivos. Então a imagem de si mesmo, o que a pessoa reconhece 
como ela mesma, é distorcida. Este descompasso nos faz sentir como que 
pilotando um cockpit desconhecido e sem manual. Ela não se sente mais 
familiar nem com a própria mente.\\ 
\indent O equívoco, o atrito, não é o problema, depois vira epifania; o 
desemparelhamento momentâneo é que pode causar avarias e degringolar em 
desespero. Este desarranjo nos faz sentir como que de volta num corpo 
bebê recém parido, desconcertado com as luzes intensas e a respiração 
seca. Alguns engatinham, outros choram.\\* 
É necessário um certo desapego, perceber que tudo isso não passa de uma 
voltinha. Evitar a repetição desesperada das mesmas atitudes falidas e 
insensatas, a paranoia delirante de querer que o mundo sempre esteja do 
jeito que estávamos acostumados, que ele reaja sempre da mesma maneira 
que achamos \emph{normal e justo}. A pessoa em bad não está sabendo 
lidar com a situação, caiu em armadilhas que ela mesma montou em sua 
mente e não consegue sair. Ele está enfrentando de frente o problema de 
viver numa realidade onde os recurso que carrega podem simplesmente não 
funcionar e precisam ser reinventados. 
\end{multicols} 
\section*{\begin{flushright}Contornando e Pendurando\end{flushright}} 
\begin{multicols}{2}
Por algum motivo está seguro que é melhor intervir em algo, que tem 
disposição e capacidade pra isso. Talvez tenha visto algum perigo que os 
outros não perceberam, talvez seja fácil prestar algum auxílio sem 
grandes chances de perturbar. Porém, lembre-se: O melhor pra geral é o 
melhor, na média, pra cada um.\\  
Parte do processo de convencer a pessoa de que está tudo bem é tratá-la 
como se estivesse tudo bem. Tente sempre localizar a pessoa, dizendo o 
que ela usou, lembrando onde ela está. Sem afobá-la, apenas 
situando-a.\\  
\indent É incomum lidarmos com situações excepcionais, não temos essa 
malemolência, principalmente naquelas em que até as estratégicas básicas 
de enfrentamento precisam ser depuradas. Alterando a percepção você não 
só altera o recorte do mundo externo que é feito.\\ 
Cuide dos pés, dê apoio e prática.\\ 
\end{multicols}
%
%
%
\end{document}
%
%
%
% vim comand autoload
% vim:set noai nocin nosi list tw=72 cc=72 wm=3 fo=tcwq fdm=marker: 
